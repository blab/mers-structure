\documentclass[stdletter,letterpaper,addrfromright,orderfromdateto,dateleft,11pt,noaddrto,sigleft]{newlfm}
\topmarginskip{-0.4in}
\bottommarginskip{-1.5in}
\leftmarginsize{1in}
\rightmarginsize{1.25in}
\sigskipbefore{0.2in}
\sigskipafter{0in}
\noLines
\nolines
\noHeadline
\noheadline
\signature{Gytis Dudas}

\namefrom{}
\addrfrom{Vaccines and Infectious Disease Division \\ Fred Hutchinson Cancer Research Center \\ 1100 Fairview Ave N \\ Seattle \\ USA}
\emailfrom{gdudas@fredhutch.org}

\greetto{Dear Editor,}
\closeline{Sincerely,}

% comments
\usepackage{color}
\usepackage{ulem}
\definecolor{purple}{rgb}{0.459,0.109,0.538}
\def\tb#1#2{\sout{#1} \textcolor{purple}{#2}}
\def\tbc#1{\textcolor{purple}{[#1]}}

\begin{document}

\begin{newlfm}
Please find attached our manuscript entitled ``Structured population models demystify MERS-CoV epidemiology''.
We would be grateful if you considered it for publication in \textit{eLife}.
Our study uses phylogenetic models that explicitly model population structure applied to Middle East Respiratory Syndrome Coronavirus (MERS-CoV) in order to quantify and partition its evolution between two major known hosts -- humans and camels. %LMC: this sentence is oddly located, because the ``solution'' comes before the ``problem'', so the reader has no idea why these fancy phylogenetic models are useful...
Since its discovery in 2012 the virus has caused over 2000 cases of MERS in humans, largely restricted to the Arabian Peninsula.
Although contact with camels, the presumed reservoir, is a known risk factor in MERS infection, the epidemiology of the virus has remained obscure.
Studies published to date have tried quantifying relative contributions of cross-species and human-to-human transmission in continuation of human MERS outbreaks, but relied heavily on either case data where case clustering is difficult to ascertain or sequence data sampled overwhelmingly from humans.
As a result these studies have reached markedly different conclusions, often at odds with established epidemiological facts.

In our manuscript we employ a structured coalescent model that quantifies patterns of diversity in MERS-CoV sequences (human or camel) by modeling host-specific evolution and cross-species transmission explicitly.
The dynamics of MERS-CoV we infer are characterised by long term viral evolution occurring exclusively in camels and human outbreaks driven largely by cross-species introductions, rather than extensive human-to-human (\textit{i.e.} community) transmission.
This explains why the continued trickle of MERS cases has been limited to the Arabian Peninsula, with little evidence of widespread human exposure across the region.
Our findings point towards the central role of the camel reservoir and a minimal, but not negligible, contribution of community transmission to continuation of MERS outbreaks in the region.
We go a step further and use Monte Carlo simulations to integrate MERS case data with structured coalescent inference to quantify the basic reproductive number ($R_{0}$) of MERS-CoV in humans from inferred human sequence clusters.
Our main finding is that MERS-CoV is relatively poor at human-to-human transmission ($R_{0}<1.0$) and that correspondingly, there have been hundreds of zoonotic introductions of the virus into humans, mostly limited to primary cases, and that MERS-CoV sequencing efforts have been uneven owing to the tendency to sequence from large MERS outbreaks in hospitals.
Finally, by leveraging additional features of the structured coalescent we show that cross-species jumps into humans appear to be seasonal, which we attribute to seasonal camel breeding patterns that result in varied force of infection.
 
We believe our manuscript joins a number of studies bridging phylogenetics and epidemiology and shows how methods common to both can be utilised synergistically to understand the drivers of transmission dynamics in rapidly evolving pathogens.
We expect the manuscript to be of great interest to epidemiologists, healthcare professionals, decision makers and colleagues alike.
Our findings clearly summarise MERS-CoV epidemiology from sequence data alone and help contextualise previous studies, rather than add to the perceived mystery of MERS-CoV.
The approaches we employ serve to both highlight the power of population structure modeling, as well as pushing the field towards greater integration of case and sequence data.

\end{newlfm}

\end{document}
