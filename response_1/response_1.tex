\documentclass[11pt,oneside,letterpaper]{article}

% graphicx package, useful for including eps and pdf graphics
\usepackage{graphicx}
\DeclareGraphicsExtensions{.pdf,.png,.jpg}

% basic packages
\usepackage{color}
\usepackage{parskip}
\usepackage{float}
\usepackage{hyperref}

% text layout
\usepackage{geometry}
\geometry{textwidth=15.25cm} % 15.25cm for single-space, 16.25cm for double-space
\geometry{textheight=22cm} % 22cm for single-space, 22.5cm for double-space

% helps to keep figures from being orphaned on a page by themselves
\renewcommand{\topfraction}{0.85}
\renewcommand{\textfraction}{0.1}

% bold the 'Figure #' in the caption and separate it with a period
% Captions will be left justified
\usepackage[labelfont=bf,labelsep=period,font=small]{caption}

% review layout with double-spacing
%\usepackage{setspace}
%\doublespacing
%\captionsetup{labelfont=bf,labelsep=period,font=doublespacing}

% cite package, to clean up citations in the main text. Do not remove.
\usepackage{cite}
%\renewcommand\citeleft{(}
%\renewcommand\citeright{)}
%\renewcommand\citeform[1]{\textsl{#1}}

% Remove brackets from numbering in list of References
\renewcommand\refname{\large References}
\makeatletter
\renewcommand{\@biblabel}[1]{\quad#1.}
\makeatother

\usepackage{authblk}
\renewcommand\Authands{ \& }
\renewcommand\Authfont{\normalsize \bf}
\renewcommand\Affilfont{\small \normalfont}
\makeatletter
\renewcommand\AB@affilsepx{, \protect\Affilfont}
\makeatother

% notation
\usepackage{amsmath}
\usepackage{amssymb}


\begin{document}

\newgeometry{top=4cm}

Dear eLife editorial board,

Thank you for your thorough and insightful comments. Please find attached our revised manuscript entitled ``MERS-CoV spillover at the camel-human interface''.  This is an update to submission 14-08-2017-RA-eLife-31257. The editorial assessment identified five essential revisions, in addition to minor points raised by individual reviewers.

Point-by-point review responses are attached.

Sincerely,\\
Gytis Dudas

\newpage

\section*{Reviewer responses}

Original comments are in plain text.  Our responses follow in \textbf{bold}.

\section*{Essential revisions}
1. The population genetic model (the particular form of structured coalescent) is highly idealised and this may influence the quantitative conclusions, although we suspect the conclusions are quite robust qualitatively. This model specifically estimates the rate of a lineage moving between demes going backwards in time; the numbers cited for the camel->human rate is really the rate that a lineage in humans goes to a camel going down the tree. The relationship between these migration rates and the epidemiologically meaningful transmission rate is complex and depends among other things on the ratio of population sizes in both demes. Per-capita transmission rates could be estimated using an epidemiologically structured coalescent model (see e.g. papers by Volz \& Rasmussen), which would ideally be stochastic due to bursty dynamics in humans. But this would be a large undertaking and so we suggest that for now the distinction is clarified. Overall, a little more discussion of the complexity and pitfalls when relating idealised population genetic models (like the island model used here) to a noisy nonlinear epidemic like this one might be merited.

\textbf{Yes, we agree that the structured coalescent approach is idealised and does not reflect a meaningful rate of zoonotic transfer of lineages, which is the reason we restricted any mention of rates to supplementary figures and do not attach numbers whenever rates are mentioned, but still report the number of introductions observed in the sequence data. We have altered Figure S2 in the reviewed manuscript to reflect that the rates shown are backwards in time. We have added the following sentences to the discussion to highlight the fact that the coalescent model employed is not ideal:}

\begin{quotation}
Although we recover migration rates from our model (Figure S2), these only pertain to sequences and in no way reflect the epidemiologically relevant \textit{per capita} rates of zoonotic spillover events.
\end{quotation}

\begin{quotation}
Having said that, there are highly customisable coalescent methods available that extend the methods used here to accommodate migration rates and population sizes varying through time, and integrate alternative sources of information that are able to fit stochastic noisy nonlinear models (Rasmussen \textit{et al}., 2014) which would be more appropriate for MERS-CoV.
Some distinct aspects of MERS-CoV epidemiology could not be captured in our methodology, such as hospital outbreaks where $R_{0}$ is expected to be consistently closer to 1.0 compared to community transmission of MERS-CoV.
Outside of coalescent-based models there are population structure models that explicitly relate epidemiological parameters to the branching process observed in sequence data (K\"{u}hnert \textit{et al}., 2016), but often rely on specifying numerous informative priors and can suffer from MCMC convergence issues.
\end{quotation}

% \begin{quotation}
%
% \end{quotation}
2. p4, `Our analyses recover these results despite sequence data heavily skewed towards non-uniformly sampled human cases and are robust to choice of prior'. This is a quite nice result and raises the question if skewed sampling would bias estimates if using a substitution model approach (`discrete trait analysis', DTA). It would strengthen the paper to include a comparison of the structured coalescent estimates to another method for ancestral states; the most popular approach in beast has been substitution models (DTA). These may give divergent results because of skewed sampling. It would be rather easy for the authors to run a DTA and if biased, this would serve as a good cautionary example when sampling is highly skewed towards one deme.

\textbf{An excellent suggestion, thank you. We have run this analysis and include the results as a new supplementary figure S3. As expected, the skewed sampling results in a reconstruction of ancestral states that puts humans as the source of most MERS-CoV lineages in camels. We have added the appropriate description of methods as well as the following paragraph in results:}

\begin{quotation}
Our findings suggest that instances of human infection with MERS-CoV are more common than currently thought, with exceedingly short transmission chains mostly limited to primary cases that might be mild and ultimately not detected by surveillance or sequencing.
Structured coalescent analyses recover the camel-centered picture of MERS-CoV evolution despite sequence data heavily skewed towards non-uniformly sampled human cases and are robust to choice of prior.
Comparing these results with a currently standard discrete trait analysis (Lemey \textit{et al.}, 2009) approach for ancestral state reconstruction shows dramatic differences in host reconstruction at internal nodes (Figure S3).
Discrete trait analysis reconstruction identifies both camels and humans as important hosts for MERS-CoV persistence, but with humans as the ultimate source of camel infections.
A similar approach has been attempted previously (Zhang \textit{et al.}, 2016), but this interpretation of MERS-CoV evolution disagrees with lack of continuing human transmission chains outside of Arabian peninsula, low seroprevalence in humans and very high seroprevalence in camels across Saudi Arabia.
We suspect that this particular discrete trait analysis reconstruction is false due to biased data, \textit{i.e.} having nearly twice as many MERS-CoV sequences from humans (N=174) than from camels (N=100) and the inability of the model to account for and quantify vastly different rates of coalescence in the phylogenetic vicinity of both types of sequences.
\end{quotation}

3. A comparison to ML tree reconstruction could potentially be illuminating. We think you could be clearer about what drives the results in your paper. It is unusual for a phylogenetic ancestral reconstruction, that the results seem to be determined as much by the coalescent assumptions as by the tree topology. The two-patch model had a much higher coalescent rate in the human deme than in the camel deme - so long branches are only really possible in the camel deme. This may be why for example, staring at the top clade of figure 1, one can see camel ancestry to a whole bunch of human sequences that are not topologically separated by camel sequences. If this is correct, these results may not necessarily be wrong, but it made us slightly uncomfortable that the results are driven by the coalescent model, not the tree topology. Please elaborate, either correcting us, or explaining better. A simple test of this hypothesis would be that an ML ancestral reconstruction on the ML tree would not give the same clusters. I don't think that would make the ML result correct, but it might be an enlightening comparison. Or you may prefer another way to address this.

\textbf{A very good point. We share the suspicion that the results are largely driven by contrasts in effective population sizes between demes. In addition to the requested maximum likelihood phylogeny (supplementary Figure S14) we also ran a structured coalescent analysis where deme sizes are enforced to be the same (supplementary Figure S4). This model fails in a similar way to a DTA reconstruction shown in the new Figure S3. We now explain how the structured coalescent arrives at the tree shown in Figure 1 in the discussion:}

\begin{quotation}
When allowed different deme-specific population sizes, the structured coalescent model succeeds because a large proportion of human sequences fall into tightly connected clusters, which informs a low estimate for the population size of the human deme. 
This in turn, informs the inferred state of long ancestral branches in the phylogeny, \textit{i.e.} because these long branches are not immediately coalescing, they are most likely in camels.
\end{quotation}

4. Easily addressed, but important. The paper already sounds a strong voice of concern in the final paragraph, but we think this could be even stronger. Antia et al Nature 2003 first showed, using a simple branching process, that for most genetic landscapes, the probability of a pathogen evolving to state with R0>1 increases dramatically as a function of the wild-type R0. So R0~0.8 is much worse than R0~0.3. More sophisticated models have been done since, especially by Llyod-smith's group, but the basic result is sound. In the light of this theoretical work, your findings are not at all reassuring.

\textbf{We agree that this is an important point, but also feel that it is difficult to formulate warnings about pandemic potential without overstating the case. We also believe that adaptive landscapes play a considerable role in emerging pandemics. We added an additional sentence to the discussion and refer to the Antia et al study:}

\begin{quotation}
Previous modeling studies (Antia \textit{et al}., 2003; Part \textit{et al.}, 2013) suggest a positive relationship between initial $R_{0}$ in the human host and probability of evolutionary emergence of a novel strain which passes the supercritical threshold of $R_{0}>1.0$.
This leaves MERS-CoV in a precarious position; on one hand its current $R_{0}$ of $\sim$0.7 is certainly less than 1, but its proximity to the supercritical threshold raises alarm.
With very little known about the fitness landscape or adaptive potential of MERS-CoV, it is incredibly challenging to predict the likelihood of the emergence more transmissible variants.
In light of these difficulties, we encourage continued genomic surveillance of MERS-CoV in the camel reservoir and from sporadic human cases to rapidly identify a supercritical variant, if one does emerge.
\end{quotation}

5. More generally, the model choices need better explaining. Why delve into a structured coalescent in BEAST2 for the ancestral reconstruction, but go back to the Skygrid in BEAST1 for computations of Ne? We assume this is a pragmatic choice, and for the latter you carefully reduced the human clusters to reduce bias, but we think the rationale for your choices need laying out more clearly. Even if pragmatic rather than principled, (e.g. there are no structure coalescent options in BEAST1), we think it still needs to be stated why you made the choices you did. Especially since there are other recently-developed BEAST2 packages that could be used to fit the same structured coalescent model: BASTA \& MASCOT, as well as the very flexible PhyDyn package (which might offer improvements in computation time).

\textbf{In hindsight we see how our choices may have seemed arbitrary. Indeed all cases of using BEAST 1 vs BEAST 2 came down to what models were implemented in which package. We have tried clarified our choices throughout the manuscript.}

% \begin{quotation}
%
% \end{quotation}

\section*{Reviewer \#1}

1. p3, `...population dynamics within demes...'
The models used here assumed constant effective size in each deme.

\textbf{The reviewer is absolutely correct here. This has been fixed in the text.}

2. p6-7, I found the description of the simulation study and simulation parameters difficult to follow. For example, greater clarity would be appreciated for terms like `bias level' in Figure 3.
These terms are clarified in the supporting methods, but I found the main text incomprehensible without seeing that section first. I would make an effort to transfer more detailed methods related to the branching process model here.

\textbf{We agree that the simulation study could have been explained better. We elaborated the description of simulations in results to say:}

\begin{quotation}
Sequencing simulations are performed via a multivariate hypergeometric distribution, where the probability of sequencing from a particular cluster depends on available sequencing capacity (number of trials), numbers of cases in the cluster (number of successes) and number of cases outside the cluster (number of failures).
In addition, each hypergeometric distribution used to simulate sequencing is concentrated via a bias parameter, that enriches for large sequence clusters at the expense of smaller ones.
\end{quotation}

3. p 8-11, What are the implications of recombinant MERS CoV in camels for estimation of evolutionary rates and TMRCAs?

\textbf{An excellent point. Concerns over artefacts of recombination is what prevented us from reporting on evolutionary rates and tMRCAs of interest, such as our postulated introduction of the virus into the Arabian peninsula. We added a discussion about the potential effects of recombination in the results section and report evolutionary rates recovered by two methods:}

\begin{quotation}
Overall, recombination presents a serious hurdle for detailed phylodynamic analyses in MERS-CoV, especially where lineages evolving in camels are concerned.
Amongst other parameters of interest, recombination is expected to interfere with molecular clocks, where transferred genomic regions can give the impression of branches undergoing rapid evolution, or branches where recombination results in reversions appearing to evolve slow.
In addition to its potential to influence tree topology, recombination in molecular sequence data is an erratic force with unpredictable effects.
We suspect that the effects of recombination in MERS-CoV data are reigned in by a relatively small effective population size of the virus in Saudi Arabia (see next section) where haplotypes are fixed or nearly fixed, thus preventing an accumulation of genetic diversity that would then be reshuffled via recombination.
Nevertheless, we choose not to report on any particular estimates for times of common ancestors (tMRCAs), even though these are expected to be somewhat robust for dating human clusters, and we do not report on the evolutionary rate of the virus, even though it appears to fall firmly within the expected range for RNA viruses: regression of nucleotide differences to Jordan-N3/2012 genome against sequence collection dates yields a rate of $4.59 \times 10^{-4}$ subs/site/year, Bayesian structured coalescent estimate from primary analysis $9.57 \times 10^{-4}$ (95\% HPDs: $8.28-10.9 \times 10^{-4}$) subs/site/year.
\end{quotation}

4. p 10, `...relatively low overall with a mean estimate of N of 3.49 years (95\% HPD: 2.71-4.38), and consequently MERS-CoV phylogeny resembles a ladder often seen in human influenza A virus phylogenies...' I would not say that low Ne directly implies a ladder-like tree or vice versa, although it may produce such a tree in combination with highly dispersed sample times.

\textbf{We agree that casually calling the effective population size `low' does not explain the shape of the MERS-CoV phylogeny. We have rephrased this sentence to emphasise that the rate of coalescence observed in that particular part of the MERS-CoV phylogeny is high compared to the period over which sampling took place:}

\begin{quotation}
However, we do note that coalescence rate estimates are high relative to the sampling time period, with a mean estimate of $N_e\tau$ at 3.49 years (95\% HPD: 2.71--4.38), and consequently MERS-CoV phylogeny resembles a ladder often seen in human influenza A virus phylogenies.
\end{quotation}

5. p 11, I think these back-of-the-envelope calculations to relate Ne to epidemic size are illuminating, however I would do some things differently:

5.1. A 10 day generation interval is assumed based on a 20 day infectious period, however this is strongly dependent on assumptions about how infectiousness varies over the course of infection. Note that I prefer the term `generation interval' for this particular situation, see eg https://www.ncbi.nlm.nih.gov/pmc/articles/PMC2365921/. In a simple SIR model, the mean generation interval will equal the mean infectious period, and the choice of 10 days seems a little strange. There is certainly considerable uncertainty about parameters here, so at least the authors could compute this over a range of values.

\textbf{}

5.2. Other theoretical work suggests a slightly different relationship between Ne tau and I(t), specifically that it will depend more on transmission rate than generation time. However if making the approximation that I(t) is constant (which the authors are) , the formulas will be the same up to a factor of 2x. See Koelle, 2011, DOI: 10.1098/rsif.2011.0495 \& Volz and Frost 2013, DOI: 10.1098/rstb.2010.0060. Specifically, I would use (Ne tau ) = I(t) / 2 * [per capita transmission rate], but some of the formulas in Koelle's paper may be more realistic for MERS.
and equating beta = 1 / [generation time] gives 2x the authors' estimate if solving for I.
Note that if using a 20 day generation interval and the above formula, you would have I = 127, so this doesn't make any practical difference, but is more elegant in my opinion, since we are exact about the epidemic model that the estimate is based on. If some estimate of the dispersion parameter for R0 could be obtained from the cluster size distribution, a better estimate for I could be obtained using equations in Koelle 2011.

\textbf{}

6. P 15+, I identify the following weaknesses in the simulation study, and I would candidly discuss these in the Discussion section:

6.1. Dispersion parameter is set at 0.1 based on a previous branching process study, however there's not great support for re-using the value for a different epidemic. Was an effort made to estimate this parameter?

\textbf{The dispersion parameter was indeed chosen from a previous study. We have run small scale analyses over a grid of three bias levels (same as the main analyses), 23 levels of $R_{0}$ (0.50--1.05), and five levels of dispersion (0.004, 0.02, 0.10, 0.50 and 1.0) with 2000 replicate simulations for each combination of variables. Numbers of simulations matching empirical data are shown in Figure S15.}

6.2. The sampling model accounts for `sequencing capacity', which is good. But is it the case that sequencing probability varied over the course of the epidemic?

\textbf{A very good point. We do not expect `sequencing capacity' to have been good in the initial stages of the outbreak and perhaps even to have peaked and declined at various stages, \textit{e.g.} during the Korean MERS outbreak. We think that addressing this analytically is perhaps a dimension too far, especially given the number of possible arrangements of sequencing intensity. We have added the following sentence to the results section:}

\begin{quotation}
Note that this ``sequencing capacity'' parameter does not vary over time, even though MERS-CoV sequencing efforts have varied in intensity, starting out slow due to lack of awareness, methods and materials and increasing in response to hospital outbreaks later.
\end{quotation}

6.3. The `model fitting' methods are ad-hoc. This is a situation where approximate Bayes computation would be appropriate. But I would try GEE/method of moments. I suspect the first three standardized moments of cluster sizes would be more useful than using mean/median/sd. - The moments could also be computed analytically if you dispense with the `sequencing capacity' aspect of the sampling model, so would be very fast.

\textbf{Our approach is indeed somewhat ad-hoc, though closely related to ABC. We have replaced the tolerance parameter of ABC with sequence data uncertainty and otherwise treat parameters of interest as being under uniform priors. We have followed the suggestion here to switch from mean/median/stdev to the first three standardised moments of sequence cluster size and found the results to be virtually identical. We do see somewhat better separation of simulations when plotting mean sequence cluster size versus their skewness in all figures summarising the results of Monte Carlo simulations.}

6.4. Can you show a comparison of the empirical distribution of cluster sizes and the fitted distribution? A QQ plot? Are there any aspects of the empirical distribution that are not well characterised by the branching process model, eg at the tails?

\textbf{We have generated QQ plots of sequence cluster size distribution observed across the posterior distribution of structured coalescent trees (picked at random) against matching sequence cluster size distributions from the primary Monte Carlo simulation with bias parameter set to 2.0. Simulated sequence clusters tend to exhibit a left skew near the middle of the distribution, but not so much near the tails. We have added figure S9 to show this.}


\section*{Reviewer \#2}

1. Page 3 - where the authors state that ``Once human MERS-CoV infections are established, hoever, we find that MERS0CoV is poor at transmitted between humans" some references should be added.

\textbf{The sentence was meant to summarise our findings. We have changed the sentence to clarify that the statement is based on our findings:}

\begin{quotation}
However, we find that MERS-CoV, once introduced into humans, follows transient transmission chains that soon abate.
\end{quotation}

2. In discussion: Could the author speculate how more sequences from humans and animals, particularly recent ones, or sequences from other geographic areas (east africa, north africa, pakistan) would potentially impace their results?

\textbf{We agree that understanding the distribution of MERS zoonotic risk to humans is of paramout importance, but feel that such speculation is outside the remit of the current study.
During the time the study was performed (first half of 2017) human MERS-CoV sequences comprised a large majority of available MERS-CoV sequences.
A study published during the first round of review (November 2017) that contributed over 100 new MERS-CoV genomes from camels in United Arab Emirates (Yusof \textit{et al.}, 2017, doi:10.1038/emi.2017.89) did not uncover genetic diversity that had not been seen in the Arabian peninsula previously.
We are also aware of another study into camels across much of Africa.
These are all consistent with general statements we have made in the manuscript in the methods section so far, which we have expanded further in response to this request:}

\begin{quotation}
Sequences belonging to the outgroup clade where most of MERS-CoV sequences from Egypt fall were removed out of concern that MERS epidemics in Saudi Arabia and Egypt are distinct epidemics with relatively poor sampling in the latter.

Were more sequences of MERS-CoV available from other parts of Africa we speculate they would fall outside of the diversity that has been sampled in Saudi Arabia and cluster with early MERS-CoV sequences from Jordan and sequences from Egyptian camels.
However, currently there are no indications of what MERS-CoV diversity looks like in camels east of Saudi Arabia.
\end{quotation}

\textbf{We also intend to eventually update the manuscript with a structured coalescent analysis that includes all the sequence data that were made available recently.}

\section*{Reviewer \#3}

1. Page 2, introduction. I am not sure I would refer to the Cauchemez et al PNAS paper as traditional epidemiology. It's sophisticated statistical modelling of spatio-temporal data and the partial observation process. I think you are using different tools in epidemiology, obtaining complimentary insights, neither particularly traditional. You could just delete `traditional' or use `case-based'.

\textbf{Agreed. We have either removed or changed the three instances of `traditional' as requested.}

2. In the same paragraph the sentence starting `Genetic epidemiology' uses poor language (who is it who has shown?)

\textbf{We apologise for the poor choice of words. We have changed the sentence to say:}

\begin{quotation}
Where sequence data are relatively cheap to produce, genomic epidemiological approaches can fill this critical gap in outbreak scenarios.
\end{quotation}

3. Same page, why make a controversial statement about sequencing standing in for diagnostics? Really?

\textbf{Apologies for that sentence as well. We have removed that part of the sentence altogether.}

4. Page 5 `R0 more familiar to epidemiologists' More familiar than what? There are no previous quantitative estimates presented. Rephrase.

\textbf{Apologies, we meant to say that $R_{0}$ is a more intuitive measure of epidemic behaviour than extrapolating risk from subtrees of a phylogeny. We have changed the sentence to say:}

\begin{quotation}
However, we wanted to translate this observation into an estimate of the basic reproductive number $R_{0}$ to provide an intuitive epidemic behaviour metric that does not require expertise in reading phylogenies.
\end{quotation}

5. Page 5. Your summary statistics are mean, median and standard deviation of cluster size. It seems odd not to have a statistic measuring skewness, given how skew the cluster sizes seem to be. Why did you not use one? Do you think you would have got similar results with one?

\textbf{Thank you for the comment. It was our thinking that mean, median and standard deviation of cluster sizes would be able to capture both the overall magnitude and skew of the empirical distribution. We also accidentally omitted the number of clusters as a fourth parameter by which we filter simulated data. In retrospect we should have been using skewness from the beginning. We have since fixed both these issues and now use mean, standard deviation, skewness and number of clusters as filters for simulated data throughout the manuscript. This does not seem affect any of our results.}

6. Page 8. You refer repeatedly to spatial statistics to refer to location on genome. I think you should refer to clustering on the location on the genome.

\textbf{Agree. We have changed references to `spatial' when referring to genome positions to `clustering':}

\begin{quotation}
Another hallmark of recombination is clustering of alleles along the genome, due to how template switching, the primary mechanism of recombination in RNA viruses, occurs.
3Seq relies on the clustering of nucleotide similarities along the genome between sequence triplets -- two potential parent-donors and one candidate offspring-recipient sequences.
\end{quotation}

7. Page 9. I think the section on recombination is verbose, and potentially misleading. You didn't perform an ARG, you performed a perfectly reasonable sensitivity analysis to different partitions of the genome. I would be more upfront about this, and try to explain it more simply and clearly.

\textbf{We have tried our best to clean up the section on recombination.}

8. Discussion. I think the results of this analysis on R0 and cluster size distribution could be compared with Cauchemez et al PNAS 2016, who estimated many of the similar quantities using spatiotemporal data to define clusters. An important result of that paper was that the cluster size distributions were very skew - I think it would be good to look at skewness here.

\textbf{An excellent suggestion, however the authors of the Cauchemez \textit{et al}. study did not share any data beyond the raw input file they used. Since the actual numbers in Fig 1C in the manuscript are not reported and a publicly available implementation of the MCMC algorithm do not exist we have included an additional figure that visualises the distribution of cases recovered by simulation. These exhibit a heavy left skew, much like the distribution in Cauchemez \textit{et al}.}

% \bibliography{./../mers-structure}

\end{document}
